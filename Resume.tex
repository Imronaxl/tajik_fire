\documentclass[11pt,a4paper]{article}
\usepackage[a4paper, margin=1in]{geometry}
\usepackage[T2A]{fontenc}
\usepackage[utf8]{inputenc}
\usepackage[russian]{babel}
\usepackage[hidelinks]{hyperref}
\usepackage{enumitem}
\usepackage{titlesec}

% ================ FORMATTING ================
\setlength{\parindent}{0pt}
\setlist[itemize]{leftmargin=*, noitemsep, topsep=2pt}
\titleformat{\section}{\large\bfseries\scshape}{}{0em}{\uppercase}[\titlerule]
\titlespacing*{\section}{0pt}{8pt}{4pt}

% ================ HEADER ================
\begin{document}
{\LARGE\textbf{Гулахмадзода Имрон Бехруз}}\\[3pt]
\small
\begin{tabular}{@{}ll}
Телефон: +7 953 150-16-83 & Email: \href{mailto:gimron34@gmail.com}{gimron34@gmail.com} \\
Telegram: \href{https://t.me/Imeon_AXL}{@Imeon\_AXL} & GitHub: \href{https://github.com/tajik-fire}{github.com/tajik-fire}
\end{tabular}

% ================ SECTIONS ================
\section{Резюме}
Студент-программист с фокусом на backend-разработку и глубоким интересом к созданию высоконагруженных и безопасных систем. Автор образовательной платформы, использующей очереди задач и изоляцию кода. Ищу стажировку для применения навыков в Python, Java и распределенных системах в коммерческих проектах.

\section{Навыки}
\begin{tabular}{@{}p{0.2\textwidth} p{0.75\textwidth}}
\textbf{Языки:}         & Python, C++, Java \\
\textbf{Backend:}       & REST API, многопоточность, очереди задач (RabbitMQ/Redis), Java Servlets, JSP, WildFly, Apache HTTPD, FastCGI \\
\textbf{Базы данных:}   & PostgreSQL, SQL, SQLAlchemy (основы) \\
\textbf{Инструменты:}   & Git, Docker (основы), Linux, VS Code, GitHub Actions \\
\textbf{Soft Skills:}   & Командная работа, планирование задач, решение сложных проблем, самообучение
\end{tabular}

\section{Опыт и проекты}
\textbf{Tajik-Fire | Образовательная платформа} \hfill 2023 -- н.в.\\
\textit{Инициатор и Backend-разработчик}
\begin{itemize}
\item Спроектировал и реализовал серверную архитектуру с системой асинхронных очередей для проверки решений, что позволило обрабатывать задачи множества пользователей одновременно.
\item Внедрил механизм безопасного выполнения пользовательского кода с помощью изолированных процессов, ограничений ресурсов и таймаутов, полностью исключив риски перегрузки сервера.
\item Участвовал в UX-тестировании и на основе обратной связи улучшил интерфейс, повысив удобство использования платформы.
\item \textbf{Стек:} Python, REST, многопоточность, изоляция процессов, Linux.
\item \href{https://github.com/tajik-fire/tajik-fire.github.io}{Репозиторий проекта}
\end{itemize}

\textbf{Веб-приложение для проверки попадания точек в область} \hfill 2024\\
\textit{Backend-разработчик (учебный проект)}
\begin{itemize}
\item Разработал FastCGI-сервер на Java, обрабатывающий HTTP-запросы и валидирующий данные.
\item Реализовал сохранение состояния сессии между запросами, обеспечив непрерывность работы пользователя.
\item \textbf{Стек:} Java, FastCGI, Apache HTTPD, AJAX, JSON.
\end{itemize}

\textbf{Веб-приложение на Java Servlets (MVC)} \hfill 2024\\
\textit{Full-stack разработчик (учебный проект)}
\begin{itemize}
\item Создал приложение по шаблону MVC с использованием сервлетов (контроллер) и JSP (представление).
\item Настроил развертывание на сервере приложений WildFly, обеспечив корректную работу в production-подобной среде.
\item \textbf{Стек:} Java, Servlets, JSP, MVC, WildFly, jQuery.
\item \href{https://github.com/Imronaxl/tajik_fire}{Репозиторий проекта}
\end{itemize}

\section{Образование}
\textbf{Университет ИТМО}, Санкт-Петербург \\
Бакалавриат «Программная инженерия и компьютерные технологии» \hfill 2024 -- 2028

\section{Достижения}
\begin{itemize}
\item Финалист Баттла вузов от Яндекса \hfill 2025
\item Участник 1/4 финала ICPC и регулярных командных тренировок (BSUIR Open) \hfill 2025--2026
\item Призер республиканских олимпиад по программированию (Таджикистан) \hfill 2021--2023
\end{itemize}

\end{document}
